        \section{Grover's Algorithm}\label{chapGrover}
        Searching through an unstructured database is a task classically achieved by exhaustively evaluating every element in the database. Assume there exists a black box (oracle) that can be asked to find out if two elements are equal. Since we're looking for a specific element in a database of size N, we'd have to query the oracle on average $\frac{N}{2}$ times, or in the worst case $N$ times.\par
        Grover's algorithm, presented in \cite{grover1996}, comes as a quantum alternative to this type of problems, taking advantage of superposition by increasing desirable states' amplitudes through a process called \textit{amplitude amplification}. This method has a quadratic gain over the classical counterpart \cite{boyer1996}, being able to find a target element with $\mathcal{O}(\sqrt{N})$ oracle complexity.\par
        Let us now expand on the inner workings of the black box. We start by focusing on searching indexes     instead of directly evaluating the element and we assume $N=2^{n}$, $n$ being a positive integer. We can now define a function $f : \{0,1,...,N-1\}$ that returns $1$ when evaluating the desired (marked) element and $0$ otherwise. Since this function is to be applied to a quantum system, we must build a unitary operator $\mathcal{F}$
        \begin{equation}
                        \mathcal{F}\ket{x}\ket{i} = \ket{x}\ket{i\oplus f(x)} .
        \end{equation}
        where $\ket{x}$ is the index register, $\oplus$ is the binary sum operation and $\ket{i}$ is a qubit that is flipped if $f(x)=1$.\par 
        The action of the oracle on state $\ket{0}$ will be
        \begin{equation}
                \mathcal{F}\ket{x}\ket{0} = \begin{cases} \ket{x_0}\ket{1}, & \mbox{if } x = x_0 \\ \ket{x}\ket{0}, & \mbox{otherwise.} \end{cases}
        \end{equation}
        where $x_0$ is the marked element. More generically, $\mathcal{F}$ can be written as
        \begin{equation}
                        \mathcal{F}\ket{x} = (-1)^{f(x)}\ket{x} .
        \end{equation}\par
        This offers a bit of insight into the oracle, it \textit{marks} the solutions to the search problem by applying a phase shift to the solutions. 
        %Tight bounds on quantum searching
        As shown in \cite{BBHT96}, for an $N$ item search with $K$ solutions, we must only apply the oracle $\mathcal{O}(\sqrt{N})$ times, which will be useful later when we describe the \textit{single shot Grover}.\par
        The question now is, what is the procedure that determines a solution $x_0$ using $\mathcal{F}$ the minimum number of times? The answer lies in the amplitude amplification section of Grover's search, starting with the creation of a uniform superposition
                     \begin{equation}
                         \ket{\Psi} = H^{\otimes n}\ket{x} = \frac{1}{\sqrt{N}}\sum_{x=0}^{N-1} \ket{x}\tab.
                     \end{equation}
         where $H^{\otimes n}$ is the \textit{Hadamard} operator applied to an arbitrary number of states.\par
         If we were to measure $\ket{x}$ at this point, the superposition would collapse to any of the base states with the same probability $\frac{1}{N} = \frac{1}{2^n}$, which means that on average, we'd need to try $N = 2^n$ times to guess the correct item. 
         This is where amplitude amplification comes into effect, by means of a second unitary operator
                    \begin{equation}
                         \mathcal{D} = (2\ket{\Psi}\bra{\Psi} - I) = H^{\otimes n}(2\ket{0}\bra{0} - I)H^{\otimes n}   
                      \end{equation}
        
          This operator applies a conditional phase shift, with every computational basis state except $\ket{0}$ receiving a phase shift. This can also be described as the \textit{inversion about the mean}, for a state of arbitrary amplitudes
                    \begin{equation}
                        \ket{\phi} = \sum_{k=0}^{N-1} \alpha_k\ket{k}
                    \end{equation}
         the action of \mathcal{D} on state $\phi$ will be
                     \begin{equation}
                         \mathcal{D}\ket{\phi} = \sum_{k=0}^{N-1}(-\alpha_k + 2\langle \alpha \rangle)\ket{k}
                    \end{equation}
         where $\langle \alpha \rangle$ is the average of $\alpha_k$
                    \begin{equation}
                        \langle \alpha \rangle = \frac{1}{N} \sum_{k=0}^{N-1} \alpha_k\ket{k}
                    \end{equation}
                    \par
        The evolution operator that performs one step of the algorithm is then
                    \begin{equation}
                        \mathcal{U} = \mathcal{D}\mathcal{F}
                    \end{equation}
         and after $t$ steps the state of the system is
                    \begin{equation}
                        \ket{\Psi(t)} = \mathcal{U}^t\ket{\Psi}.
                    \end{equation}
        The optimal number of steps is, as aforementioned, $\sqrt{\frac{N}{K}}$ where $K$ is the number of solutions to the problem.
